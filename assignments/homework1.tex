% Created 2012-09-01 Sat 08:34
\documentclass[11pt]{article}
\usepackage[latin1]{inputenc}
\usepackage{fixltx2e}
\usepackage{url}
\usepackage{graphicx}
\usepackage{minted}
\usepackage{color}
\usepackage{longtable}
\usepackage{float}
\usepackage{wrapfig}
\usepackage{soul}
\usepackage{textcomp}
\usepackage{amsmath}
\usepackage{marvosym}
\usepackage{wasysym}
\usepackage{latexsym}
\usepackage{amssymb}
\usepackage[linktocpage,
  pdfstartview=FitH,
  colorlinks,
  linkcolor=blue,
  anchorcolor=blue,
  citecolor=blue,
  filecolor=blue,
  menucolor=blue,
  urlcolor=blue]{hyperref}
\usepackage{attachfile}
\tolerance=1000
\providecommand{\alert}[1]{\textbf{#1}}

\title{Homework 1 - Due 9/6/2012}
\author{John Kitchin}
\date{2012-09-01 Sat}
\hypersetup{
  pdfkeywords={},
  pdfsubject={},
  pdfcreator={Emacs Org-mode version 7.8.11}}

\begin{document}

\maketitle



Instructions: All calculations should be performed in python. You should turn in the code used, and the answers you got.

\section{Signup for an account at gitHub.}
\label{sec-1}

Print your username here:

Set yourself up to \href{http://www.quora.com/GitHub/What-does-it-mean-to-watch-in-GitHub}{watch}
\href{https://github.com/jkitchin/dft-course}{https://github.com/jkitchin/dft-course} and
\href{https://github.com/jkitchin/dft-book}{https://github.com/jkitchin/dft-book}.
\section{Read Chapter 1 in the text book.}
\label{sec-2}
\section{Read Section 4 in dft-book.}
\label{sec-3}

As part of this assignment, please turn in a pdf copy of dft-book that has been annotated by sticky notes using Adobe Acrobat Reader (you should be able to type Ctrl-6 to get a sticky note while the pdf is open, and then you can move it where you want and type text in it.). Please note any typos, places that are confusing, etc\ldots{}
\section{Data fitting.}
\label{sec-4}

Fit a cubic polynomial to this set of data and estimate the lattice constant that minimizes the total energy. Prepare a figure that shows the data, your fit and your estimated minimum. Hints: \texttt{numpy.polyfit}, \texttt{numpy.polyder}, \texttt{numpy.roots}, \texttt{numpy.linspace}, \texttt{numpy.polyval} will all help you do this easily.


\begin{center}
\begin{tabular}{rr}
 lattice constant ($\AA$)  &  Total Energy (eV)  \\
\hline
                      3.5  &          -3.649238  \\
                     3.55  &          -3.696204  \\
                      3.6  &          -3.719946  \\
                     3.65  &          -3.723951  \\
                      3.7  &          -3.711284  \\
                     3.75  &           -3.68426  \\
\end{tabular}
\end{center}
\section{Nonlinear algebra}
\label{sec-5}

Solve this equation: $\sin(x^2) = 0.5$ for $x$. Prepare a plot of the function and show where your solution is. Hint: \texttt{scipy.optimize.fsolve}
\section{Linear algebra}
\label{sec-6}

Solve these equations using python and linear algebra:

\begin{eqnarray}
a0 - 3 a1 + 9 a2 - 27 a3 = -2 \\
a0 - a1 + a2 - a3 = 2 \\
a0 + a1 + a2 + a3 = 5 \\
a0 + 2a1 + 4 a2 + 8 a3 = 1
\end{eqnarray}

Use linear algebra to verify your solution. Hint: see \texttt{numpy.linalg}, \texttt{numpy.dot}.

\end{document}
